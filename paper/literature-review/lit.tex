% Options for packages loaded elsewhere
\PassOptionsToPackage{unicode}{hyperref}
\PassOptionsToPackage{hyphens}{url}
%
\documentclass[
]{article}
\usepackage{amsmath,amssymb}
\usepackage{lmodern}
\usepackage{iftex}
\ifPDFTeX
  \usepackage[T1]{fontenc}
  \usepackage[utf8]{inputenc}
  \usepackage{textcomp} % provide euro and other symbols
\else % if luatex or xetex
  \usepackage{unicode-math}
  \defaultfontfeatures{Scale=MatchLowercase}
  \defaultfontfeatures[\rmfamily]{Ligatures=TeX,Scale=1}
\fi
% Use upquote if available, for straight quotes in verbatim environments
\IfFileExists{upquote.sty}{\usepackage{upquote}}{}
\IfFileExists{microtype.sty}{% use microtype if available
  \usepackage[]{microtype}
  \UseMicrotypeSet[protrusion]{basicmath} % disable protrusion for tt fonts
}{}
\makeatletter
\@ifundefined{KOMAClassName}{% if non-KOMA class
  \IfFileExists{parskip.sty}{%
    \usepackage{parskip}
  }{% else
    \setlength{\parindent}{0pt}
    \setlength{\parskip}{6pt plus 2pt minus 1pt}}
}{% if KOMA class
  \KOMAoptions{parskip=half}}
\makeatother
\usepackage{xcolor}
\IfFileExists{xurl.sty}{\usepackage{xurl}}{} % add URL line breaks if available
\IfFileExists{bookmark.sty}{\usepackage{bookmark}}{\usepackage{hyperref}}
\hypersetup{
  pdftitle={Literature Review},
  pdfauthor={Sebastian Barrios},
  hidelinks,
  pdfcreator={LaTeX via pandoc}}
\urlstyle{same} % disable monospaced font for URLs
\usepackage{longtable,booktabs,array}
\usepackage{calc} % for calculating minipage widths
% Correct order of tables after \paragraph or \subparagraph
\usepackage{etoolbox}
\makeatletter
\patchcmd\longtable{\par}{\if@noskipsec\mbox{}\fi\par}{}{}
\makeatother
% Allow footnotes in longtable head/foot
\IfFileExists{footnotehyper.sty}{\usepackage{footnotehyper}}{\usepackage{footnote}}
\makesavenoteenv{longtable}
\setlength{\emergencystretch}{3em} % prevent overfull lines
\providecommand{\tightlist}{%
  \setlength{\itemsep}{0pt}\setlength{\parskip}{0pt}}
\setcounter{secnumdepth}{-\maxdimen} % remove section numbering
\ifLuaTeX
  \usepackage{selnolig}  % disable illegal ligatures
\fi
\usepackage{longtable,booktabs,array}
\usepackage{calc} % for calculating minipage widths
% Correct order of tables after \paragraph or \subparagraph
\usepackage{etoolbox}
\makeatletter
\patchcmd\longtable{\par}{\if@noskipsec\mbox{}\fi\par}{}{}
\makeatother
% Allow footnotes in longtable head/foot
\IfFileExists{footnotehyper.sty}{\usepackage{footnotehyper}}{\usepackage{footnote}}
\makesavenoteenv{longtable}
\setlength{\emergencystretch}{3em} % prevent overfull lines
\providecommand{\tightlist}{%
  \setlength{\itemsep}{0pt}\setlength{\parskip}{0pt}}
\title{Literature Review}
\author{Sebastian Barrios}
\date{Fri Nov 19 2021}

\begin{document}
\maketitle

\hypertarget{general-problem}{%
\section{General Problem}\label{general-problem}}

Autonomous Vehicles (AVs) have been at the center of the news as the
next (or current?) big thing in the tech world. AVs are poised to
transformed our roads in safer places for everyone: passengers,
pedestrians and every other road participant. However, reaching a safety
standard where everyone feels confident with AVs has proven difficult,
as the underlying algorithms and networks have to deal with implicit
road rules, unexpected actions by road participants and a myriad of edge
cases that NNs have not been exposed to before.

One of the topics that need further work is the one that has a new level
of complexity -- the one related to the estimation of pedestrian
intentions. This complexity comes from the inherent randomness in
pedestrian behavior as individuals, and because pedestrian roads do not
have many implicit rules (such as bike lanes and others).

The papers dicussed in this Literature Review all approach this topic in
novel ways, and they all shared the same dataset, providing an
interesting view at the different ways researchers and practitioners are
looking to solve this task.

\hypertarget{concise-summary}{%
\section{Concise Summary}\label{concise-summary}}

\hypertarget{are-they-going-to-cross-a-benchmark-dataset-and-baseline-for-pedestrian-crosswalk-behavior-rasouli2017iccvw}{%
\subsection{Are they going to Cross? A Benchmark Dataset and Baseline
for Pedestrian Crosswalk Behavior
@rasouli2017ICCVW}\label{are-they-going-to-cross-a-benchmark-dataset-and-baseline-for-pedestrian-crosswalk-behavior-rasouli2017iccvw}}

The main contribution of Rasouli et al.~was the introduction of the
Joint Attention in Autonomous Driving (JAAD) dataset. In addition to
bounding boxes for pedestrian detection, the dataset also includes
pedestrian behavior and contextual annotations for each video clip. The
paper aims to prove that incorporating the latter information improves
prediction over the case of only using bounding boxes.

The dataset introduced by Rasouli et al.~includes 346 videos between 5
and 15 seconds of duration at 30 frames per second. The data was mainly
obtained in Europe (82.7\%) of the total number of video clips, and the
rest (17.3\%) was sourced in North America. All videos were recorded
with monocular cameras positioned inside the car below the rearview
mirror. These 346 comprise a total of \textasciitilde82,000 frames,
close to 2,200 pedestrians, and nearly 337,000 total bounding boxes.

Even though these videos already give researchers a new rich dataset to
perform machine learning research tasks, the uniqueness of JAAD comes
from the inclusion of behavioral tags for the labeled pedestrians (654
unique pedestrians out of the \textasciitilde2,200 samples) and the
contextual information of the scene. More specifically, the behavioral
dataset includes attributes like the state of the pedestrians before
crossing, the way the pedestrian becomes aware of the vehicle action,
and the response of the pedestrian to the action of the car (the
ego-vehicle which recorded the scene). The dataset also includes
behavioral tags for the driver (of the ego-vehicle), which include the
current state: moving slow or fast, and the response: slow down or speed
up. Information about the demographic of the pedestrians is also
included (age, gender, size of group).

\begin{longtable}[]{@{}lll@{}}
\toprule
Precondition & Attention & Response \\
\midrule
\endhead
Standing & Looking (\textgreater1 s) & Stop \\
Moving Slow & Glancing (\textless1 s) & Clear Path \\
Moving Fast & & Slow Down \\
& & Speed up \\
& & Hand Gesture \\
& & Nod \\
\bottomrule
\end{longtable}

On the other hand, contextual tags include information about the scene
where the video was recorded. The information consists of the
configuration of the road: whether it is a street (and the number of
lanes) or if it is a parking lot or a garage. It also contains
information on traffic signals (stop or pedestrian), zebra crossings,
and traffic lights.

Finally, the dataset also includes occlusion tags for the pedestrians:
partial occlusion (between 25\% and 75\% visible) and complete occlusion
(less than 25\% visible).

While using this dataset, researchers focused on determining whether
pedestrians were moving (walking) or standing and whether they performed
any attention action (i.e., looking at the oncoming car). In terms of
prediction, Rasouli et al.~reduced the problem into an image
classification problem as they were only interested in distinguishing
between 4 types of actions (walking, standing, looking towards the
traffic, and not looking.

Furthermore, to compensate for not having annotations for body parts
(i.e., a fitted skeleton for each pedestrian), the training and testing
were done by cropping the top third and bottom half from images with
minor occlusions (\textgreater=75\% visible).

Researchers found that using the AlexNet architecture over the cropped
images could achieve \textasciitilde80\% levels in average precision for
classifying whether pedestrians were looking or walking. After this
step, researchers obtained the 10-15 frames previous to the pedestrian's
decision to cross or not, and by applying the previous classification,
they reached nearly 40\% average precision on the crossing / not
crossing prediction. The notable works come from the addition of
contextual data to aid the action data. By using these two categories of
labels, researchers obtained an average precision of
\textasciitilde63\%.

\hypertarget{intention-recognition-of-pedestrians-and-cyclists-by-2d-pose-estimation}{%
\subsection{Intention Recognition of Pedestrians and Cyclists by 2D Pose
Estimation}\label{intention-recognition-of-pedestrians-and-cyclists-by-2d-pose-estimation}}

This paper focuses on anticipating the intentions of pedestrians and
cyclists -- what the researchers call ``Vulnerable Road Users'' (VRUs).
They work on pedestrian intention estimation is based on the JAAD
dataset published by Rasouli et al. @rasouli2017ICCVW, while the
cyclists estimation is done with data collected by researchers for this
experiment (I will not focus on the cyclists side of the research as it
is out of scope of my project). The researchers seek to propose a better
method to estimate VRUs intentions while relying only on monocular
2D-type frames -- in contrast with more complex approaches that involve
the use of stereo information, optical flow, or ego-motion compensation.

This work by Fang et al.~is the one that most closely resembles the
current methodology I am proposing for this project. In their
experiment, the process involves detecting and tracking pedestrians,
fitting a skeleton to each pedestrian, and then applying a
cross/no-cross classifier that relies on the fitted skeleton features.

For training the C/NC classifier, Fang et al.~used only samples with
pedestrians which bounding boxes were at least 60 pixels wide and had no
occlusions. Because they were tracking the pedestrians across the
frames, these requirements had to stay valid for more than T frames. For
tracks longer than T frames, they used a sliding window approach to
obtain different training samples. The selected value for T = 14.

The researchers showed that using the skeleton-fitting approach could
outperform models based on CNN approaches (based on AlexNet such as
@rasouli2017ICCVW), using values of T=14 and T=1), and obtain an
accuracy of \textasciitilde80\%. On testing, the model was able to
obtain a prediction after tracking a pedestrian over eight frames
(around 250ms in the JAAD dataset).

\hypertarget{predicting-intentions-of-pedestrians-from-2d-skeletal-pose-sequences-with-a-representation-focused-multi-branch-deep-learning-network}{%
\subsection{Predicting Intentions of Pedestrians from 2D Skeletal Pose
Sequences with a Representation-Focused Multi-Branch Deep Learning
Network}\label{predicting-intentions-of-pedestrians-from-2d-skeletal-pose-sequences-with-a-representation-focused-multi-branch-deep-learning-network}}

In this paper, researchers took a more complex approach to estimating
pedestrians' intentions and finally sought to propose a new network
(Skeleton-based Pedestrian Intention network, or SPI-Net) for
pedestrians' discrete intentions. The data also comes from the JAAD
dataset, and it is transformed in a similar way to the work done by Fang
et al.~by fitting a skeleton representation to each pedestrian, this
time by using the Cascaded Pyramid Network (CPN) algorithm. With this
method, researchers are able to obtain 14 different key points from each
skeleton.

On the network architecture, Gesnouin et al.~designed it as a two-branch
system. One branch focuses on Euclidean distances between the key points
of the skeleton and their evolution over time, while the other one
tracks the position of the key points over time in a Cartesian plane.

More specifically, the pedestrian pose sequence was obtained by using a
sliding window to maintain a fixed size of a 3D tensor, where the shape
is represented by ((T, K, d)) and where (T = 300,, K = 14,, d=2), for
(T) representing a fixed number of frames, (K) representing the
different key points of the skeleton and (d) representing the
coordinates of each key point.

Researchers then trained an auto-encoder for the Euclidean distance
branch (the Geometric Features branch) and concatenated the encoder part
of the auto-encoder with the CNN network trained for the Cartesian plane
branch (the Cartesian Features branch) deprived of the output layer.

With the method introduced on SPI-Net together with CPN, the experiment
reached a total accuracy of 94.4\% on the JAAD dataset.

\hypertarget{pedestrian-motion-state-estimation-from-2d-pose}{%
\subsection{Pedestrian Motion State Estimation from 2D
Pose}\label{pedestrian-motion-state-estimation-from-2d-pose}}

This paper presents a straightforward method to estimate the intention
and motion of pedestrians and other VRUs while using the same dataset as
the other experiments (the JAAD dataset). The researchers proposed a
method comprising three main steps: i) Using an existing pose estimation
software, they obtain the 2D pose of the pedestrian (in a skeleton-type
format); ii) with the key points of the skeleton obtained, they extract
high-level features, static \emph{micro motion} features, and dynamic
\emph{micro motion} features; iii) by using a seq2seq network, a model
is trained to estimate probabilities of different motions states for
each pedestrian using a softmax classifier.

Researchers refer to \emph{micro motion} features as the features
related to the movement of the joints (or key points) of the
pedestrian-related to the torso. Static micro motions include the
normalized position of key points, the normalized distance between the
different key points, and the direction of different joints. As the
paper notes \emph{Dynamic micro motion features are differential
calculations of static micro motion features in the time domain, mainly
consisting of dynamic Euclidean distance features and dynamic angle
features}.

Li et al.~then divided the micro-motion features into four distinct
groups: position features, distance features, static angle features, and
dynamic angle features. Each of these features is then encoded
separately in embedding layers and then concatenated together. For
modeling the sequence data, researchers used gated recurrent units
(GRUs), controlling the info that the model received at each time step.

The success of this experiment was measured against the original JAAD
dataset paper @rasouli2017ICCVW, and the researchers note that by using
their proposed method, they were able to improve the precision of the
predictions by 11.6\%

\hypertarget{compare-and-contrast}{%
\section{Compare and Contrast}\label{compare-and-contrast}}

The most notable commonality between all these papers is the use of the
same dataset to answer the question about whether a pedestrian (or a set
of pedestrians) intends to cross a street or intersect with the
ego-vehicle. By sharing the use of the JAAD dataset, it is relatively
straightforward to compare the best approach, or set of approaches, for
this task.

Because the nature of the task is the one with recognizing an intention
by estimating the current pose of a pedestrian, we are dealing with a
set of assumptions that still need to be validated, but interestingly,
all the experiments here described have taken this assumption as given.
As @Gesnouin et al.~describe it in their paper: \emph{Most of those
approaches are currently based on the assumption that one can link the
position of a pedestrian's joints previous to his action to an
intention. Consequently, the problem of a pedestrian discrete intention
prediction is, therefore, dealt with as an action recognition task prior
to the action, and the action label becomes the intention}.

Although the dataset is the same one for all the experiments, and it has
already been noted that the main difference is the one related to the
network architecture and algorithms used (which I will mention shortly),
it is interesting to note the way how each of these experiments treats
and manipulates the data, as well as how they selected which data is
actually fit for use and which is not (mainly in the context of
occlusion and whether the problem of not having 100\% visibility of the
pedestrian renders the frame unusable). Rasouli et al.~took a simple
approach and cropped each bounding box in two different areas to obtain
a focused image on the head area and another one on the area below the
torso while accepting occlusions of up to 25\%. On the other hand, the
remaining papers all used some algorithm to fit a skeleton to the
pedestrian, obtaining the position on the frame of the joints and other
key points(between 14-16 key points). Most of them also selected only
the frames where there was no occlusion at all.

Finally, on the topic of algorithms, this is where we see a range of
different approaches taken to achieve the same tasks. While Rasouli et
al.~treated the problem as an image classification problem, Fang et
al.~decided to create a dual approach using the skeleton features plus a
CNN architecture. On the other hand, Li et al.~decided to fit the
skeleton data but treated these features as two different subsets
(static and dynamic micro-motions) while leveraging the use of RNNs,
embeddings, and gated recurrent units. Finally, a more complex approach
came from the work of Gesnouin et al., where an encoding layer was
extracted from a trained auto-encoder and then merged with a CNN network
stripped of its classification layer to obtain estimations of the
intention of the pedestrians.

There is clear merit to each of these approaches, with the simplest ones
more apt (hardware- and economic- constraints-wise) to be deployed on a
large scale to AVs and other types of assisted driving systems.

\hypertarget{future-work}{%
\section{Future Work}\label{future-work}}

The preceding papers have made several contributions, and most
importantly, a new dataset published by Rasouli et al.~gave these
researchers rich information to pursue these pedestrian intention
estimation tasks. Although while most of the experiments proposed novel
ways of designing a network optimized for this task, only Rasouli et
al.~proposed utilizing contextual data to improve the prediction
algorithm. To the best of my knowledge, all the other experiments
focused on the video clips but did not take advantage of much of the
behavioral and contextual attributes that the dataset contains. In this
sense, it would be interesting to understand what kind of achievements
could we see when some of the best algorithms presented in this
literature review -- such as the one by @Gesnouin et al.~---- would
include contextual data (already knowing that the paper for the original
dataset increased their precision in 20 percentage points by including
this information.

Furthermore, most of the datasets discarded frames and pedestrians that
were at some level occluded (only Rasouli included occluded pedestrians
up to 25\%). Even though it makes sense from a training perspective, in
real life scenarios, AVs will need to be able to make predictions over
heavily occluded VRUs, as the occlusion is no excuse not to be able to
perform safety maneuvers to keep the passengers of the ego-vehicle and
other road participants safe.

Finally, Rasouli et al.~mentioned in their paper that further work needs
to be performed as dynamic factors such as velocity changes, changes in
the state of the vehicle, and pedestrian's sequences of actions were not
taken into consideration in their research or dataset. Also, more work
needs to be performed where the demographic context of the road
participants, group behavior, and weather conditions are supplied in
future models.

\end{document}
